Status 08.25:
Forsøkt å skrive et avnsitt om de såkalte "Dirichlet scale mixture" priorene. Håper det faller i smak.

Liste over ting som bør gjøres:

Dirichlet priorene virker å faktisk fungere bedre med slike interaksjoner, i allefall når N ikke er veldig stor. Prøv å simuler fra funksjonen som Geir og Aleksandir kom frem til fra Abalone datasettet, det kan gi oss mer lignende data som kan brukes for å styrke vår hypotese om at Dirichlet komponenten gjør priorene mer robuste når det er mye interaksjoner.

Vi har plot av konvergens vs feil i de ulike testpunktene i konvergens koden. Det kan være et aktuelt plot, kanskje enda bedre enn Rhat vs RMSE plottet.

Vi må finne CRPS for Friedman og Abalone dataen, selv om det er tidkrevende. Prøv å regn det ut når du ikke sampler noe annet.

En siste ting kan være å se på bias variansen, eventuelt øke den for å gjøre bias leddene mindre informative.

Vi kan vurdere å ta med SPSL og Normal-Gamma priorer, men jeg ville prøvd med Dirichlet-Laplace prioren før vi bruker de andre. Det krever i så fall et nytt Stan script.


Status 29.08.2025:
Jeg har nå skrevet inn Proposition 1 og 2 og teoremer 1 og 2, samt beviser for de alle. Det som mangler er forventning og varians for både proposition 2 og teorem 2. Se chat med ChatGPT, hvor den nyeste meldingen med A gjelder proposition 1 og B gjelder teorem 2. Jeg tror hvis vi får inn forventning og varians her så har vi kommet langt!
Jeg kunne godt tenkt meg å fått noen til å se over bevisene mine. Spesielt der hvor jeg;
-Ekspanderer til en sum
-Bytter sum og integral
-Bruker analytisk kontinuering
Dersom disse sitter tror jeg jaggu det blir artikkel.


Det som da altså må gjøres fremover, er å lage en ryddig og fin Overleaf. Først må forventning og varians for Proposition 1 og Teorem 2 inn. Etter det handler mye om å tolke resultatene og å skrive en god artikkel som forteller en historie og selger poengene godt!

En figur som ville gjort seg bra, er den som nå er av fordelingen til kappa(ligger i kappa.ipynb). Jeg tror vi her kunne testet for forskjellige frihetsgrader og verdier for a (altså konstanten foran den tilfeldige skalaen). Kanskje vi også burde simulere litt, det er litt vanskelig å vite hvordan q-leddet oppfører seg. Det går jo litt hånd i hånd med tolkningen av ting, så det sampsillet får vi vel til.

05.09.2025:

Et plot som kunne vært kult å ha med er kappa for prior og posterior. Jeg bruker Friedman med N=200 for å gjøre dette. Jeg kan bruke modellene for regresjon direkte, men må kjøre prior modellene på nytt. I tillegg må jeg regne ut q to ganger, en gang for prior modellen og en gang for posterior modellen.

Her er det som nå må gjøres:
Klassifiseringen fro Dirichlet Student T må kjøres om igjen for både Relu og Tanh, for både Moons og Breastcancer. Regresjonen for Dirichlet Student T må kjøres om igjen for både Relu og Tanh, for både Abalone og Friedman. Nå skaleres ingen med p, alle regulariserer og alle bruker tau. Vi legger det rett inn i resultatene, og oppdaterer Overleaf.

Du burde også kjøre konvergens en gang til. Gjør det for DST, og for alle modeller med N=500, men kjør kun for ett seed, trenger ikke flere. Du bør kopiere over ett av datasettene med N=500, også kjøre, også kan du bare beholde datasettet i den lille mappa til Friedman. Gjør ikke noe det.

Deretter vil jeg ha fine plot for prior/posterior. Det er egentlig på plass, men du må "rydde opp" i plottene.

Deretter kan vi se på de siste utregningene som nevnt tidligere.

08.09.2025:
Et spørsmål som jeg stiller er om skaleringen med p faktisk har vært bra? Virker som om jeg får dårligere resultater nå. Jeg tester med moons datasettet en gang til, denne gangen med skalering på DST, for å sjekke om det blir bedre. Det virker ikke som om det er noe dårligere på Breastcancer dataen. Er moons for simpel? Og dermed blir DST for komplisert?

SVAR: Med skaleringen, så får vi jevnt over bedre resultater for Relu DST på moons datasettet, men ikke særlig forskjell på breastcancer. Dette var da litt rart, men kanskje det er fordi datasettet er relativt lite og dermed blir modellen veldig sensitiv? Resultatene som nå ligger inne for relu er med skalering. 

python3 utils/run_all_classification_models.py --model dirichlet_student_t --output_dir results/classification/single_layer/relu/moons ✅

python3 utils/run_all_classification_models.py --model dirichlet_student_t_tanh --output_dir results/classification/single_layer/tanh/moons 🤷🏼‍♂️

python3 utils/run_breastcancer.py --model dirichlet_student_t --output_dir results/classification/single_layer/relu/breastcancer ✅

python3 utils/run_breastcancer.py --model dirichlet_student_t_tanh --output_dir results/classification/single_layer/tanh/breastcancer ✅

python3 utils/run_all_regression_models.py --model dirichlet_student_t --output_dir results/regression/single_layer/relu/friedman ✅

python3 utils/run_all_regression_models.py --model dirichlet_student_t_tanh --output_dir results/regression/single_layer/tanh/friedman ✅

python3 utils/run_abalone.py --model dirichlet_student_t --output_dir results/regression/single_layer/relu/abalone ✅

python3 utils/run_abalone.py --model dirichlet_student_t_tanh --output_dir results/regression/single_layer/tanh/abalone ✅

python3 utils/run_all_regression_models.py --model dirichlet_student_t --output_dir results/regression/single_layer/relu/friedman/convergence --sample 2000 --warmup 5000 --overwrite ✅

python3 utils/run_all_regression_models.py --model dirichlet_student_t_tanh --output_dir results/regression/single_layer/tanh/friedman/convergence --sample 2000 --warmup 5000 --overwrite ✅
